\documentclass[12pt]{article}

% Adjust The page size via use of the geometry package
% Paper is set to 4in by 8in. 
\usepackage[paperwidth=288pt, paperheight=576,textwidth=248pt,textheight=512]{geometry}
\usepackage[utf8]{inputenc}

% I prefer to use Open Sans which is a sans serif font.
% Sans Serif fonts are good for digital documents in general.
% If you like some other sans serif font, then include it instead
% or if you'd like to stick with the default, just comment the next
% line.
\usepackage[default]{opensans}
\usepackage[T1]{fontenc}

% Add the rest of your preamble here
\usepackage{amsmath}
\usepackage{parskip}
% etc.

\title{Sample Mobile Document}
\author{Aditya Chincholi}
\date{August 2020}

\begin{document}
% Set all the fonts to sans serif.
% Comment the next line if you like the LaTeX default font.
\renewcommand{\familydefault}{\sfdefault}

% Handle overflowing text due to size restriction by
% hyphenating words
\sloppy

% Start your document here.
\maketitle
\pagebreak

\section{Introduction}
In this article, we will attempt to showcase how easy it is to
produce a document in LateX which can be easily read on a phone
or any handheld device. It is not a surprising fact that most
people find it hard to read and navigate a LaTeX document when on
the phone. LaTeX defaults are designed for printed paper and when
viewed on phone, people run into these issues:

\begin{itemize}
	\item The large paper size leads to very small font size when
	the page is viewed as a whole in portrait mode.
	\item Landscape mode requires that the phone be held horizontally
	which is usually inconvenient and requires a lot of scrolling.
	\item Many a times, one has to zoom into a small part of the
	document and then move horizontally following the line. This
	makes navigation a headache.
\end{itemize}

So let's try and produce a version of the document suited for mobile
viewing perhaps? Reading ought to be easy and natural, the design of
the document should get out of the way and let us concentrate on the
content. Hence, I present to you this format. The template.xml file
included with this, will allow you to easily switch between compiling
for desktop and mobile viewers. How hard is it to just compile twice
and put out two files instead of one, if it affords the viewer a
better experience and correspondingly increases the chances your
document will be read properly?

Enough talk, let's see some math here.

\begin{align*}
	\int_{-\inf}^{+\inf} e^{\frac{-x^2}{2}} dx &= \sqrt{2\pi} \\
	e^{\pi i} + 1 &= 0
\end{align*}

Even this complicated equation I found by searching 'complicated latex
formula' on Google.

\begin{align*}
	S(\omega)
	&= \frac{\alpha g^2}{\omega^5} e^{[ -0.74\bigl\{\frac{\omega U_\omega 19.5}{g}\bigr\}^{\!-4}\,]} \\
	&= \frac{\alpha g^2}{\omega^5} \exp\Bigl[ -0.74\Bigl\{\frac{\omega U_\omega 19.5}{g}\Bigr\}^{\!-4}\,\Bigr] 
\end{align*}

But you might say \textit{"All this sans serif font business is making me feel
weird. I like the standard LaTeX font."}. Sure! Skip to the next page for that.

\pagebreak

\rmfamily

\section{\textrm{Introduction II}}
In this article, we will attempt to showcase how easy it is to
produce a document in LateX which can be easily read on a phone
or any handheld device. It is not a surprising fact that most
people find it hard to read and navigate a LaTeX document when on
the phone. LaTeX defaults are designed for printed paper and when
viewed on phone, people run into these issues:

\begin{itemize}
	\item The large paper size leads to very small font size when
	the page is viewed as a whole in portrait mode.
	\item Landscape mode requires that the phone be held horizontally
	which is usually inconvenient and requires a lot of scrolling.
	\item Many a times, one has to zoom into a small part of the
	document and then move horizontally following the line. This
	makes navigation a headache.
\end{itemize}

So let's try and produce a version of the document suited for mobile
viewing perhaps? Reading ought to be easy and natural, the design of
the document should get out of the way and let us concentrate on the
content. Hence, I present to you this format. The template.xml file
included with this, will allow you to easily switch between compiling
for desktop and mobile viewers. How hard is it to just compile twice
and put out two files instead of one, if it affords the viewer a
better experience and correspondingly increases the chances your
document will be read properly?

Enough talk, let's see some math here.

\begin{align*}
	\int_{-\inf}^{+\inf} e^{\frac{-x^2}{2}} dx &= \sqrt{2\pi} \\
	e^{\pi i} + 1 &= 0
\end{align*}

Even this complicated equation I found by searching 'complicated latex
formula' on Google.

\begin{align*}
	S(\omega)
	&= \frac{\alpha g^2}{\omega^5} e^{[ -0.74\bigl\{\frac{\omega U_\omega 19.5}{g}\bigr\}^{\!-4}\,]} \\
	&= \frac{\alpha g^2}{\omega^5} \exp\Bigl[ -0.74\Bigl\{\frac{\omega U_\omega 19.5}{g}\Bigr\}^{\!-4}\,\Bigr] 
\end{align*}

\end{document}

% Author: Aditya Chincholi
% Contact Me for any queries/suggestions at
% aditya.chincholi@students.iiserpune.ac.in
% In case you lost the template file, 
% you can also find this template on GitHub
% at www.github.com/adch99/

% Inspired by D P Story from acrotex.
% If you want to experiment, try using 
%\usepackage[smartphone,useforms]{aeb_mobile}
% Give http://www.acrotex.net/blog/?p=766 a look.
